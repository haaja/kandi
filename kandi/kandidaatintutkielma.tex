% --- Template for thesis / report with tktltiki2 class ---
%
% last updated 2013/02/15 for tkltiki2 v1.02

\documentclass[finnish]{tktltiki2}

% tktltiki2 automatically loads babel, so you can simply
% give the language parameter (e.g. finnish, swedish, english, british) as
% a parameter for the class: \documentclass[finnish]{tktltiki2}.
% The information on title and abstract is generated automatically depending on
% the language, see below if you need to change any of these manually.
%
% Class options:
% - grading                 -- Print labels for grading information on the front page.
% - disablelastpagecounter  -- Disables the automatic generation of page number information
%                              in the abstract. See also \numberofpagesinformation{} command below.
%
% The class also respects the following options of article class:
%   10pt, 11pt, 12pt, final, draft, oneside, twoside,
%   openright, openany, onecolumn, twocolumn, leqno, fleqn
%
% The default font size is 11pt. The paper size used is A4, other sizes are not supported.
%
% rubber: module pdftex

% --- General packages ---

\usepackage[utf8]{inputenc}
\usepackage[T1]{fontenc}
\usepackage{lmodern}
\usepackage{microtype}
\usepackage{amsfonts,amsmath,amssymb,amsthm,booktabs,color,enumitem,graphicx,cite}
\usepackage[pdftex,hidelinks]{hyperref}

% Automatically set the PDF metadata fields
\makeatletter
\AtBeginDocument{\hypersetup{pdftitle = {\@title}, pdfauthor = {\@author}}}
\makeatother

% --- Language-related settings ---
%
% these should be modified according to your language

% babelbib for non-english bibliography using bibtex
\usepackage[fixlanguage]{babelbib}
\selectbiblanguage{finnish}

% add bibliography to the table of contents
\usepackage[nottoc]{tocbibind}
% tocbibind renames the bibliography, use the following to change it back
\settocbibname{Lähteet}

% --- Theorem environment definitions ---

\newtheorem{lau}{Lause}
\newtheorem{lem}[lau]{Lemma}
\newtheorem{kor}[lau]{Korollaari}

\theoremstyle{definition}
\newtheorem{maar}[lau]{Määritelmä}
\newtheorem{ong}{Ongelma}
\newtheorem{alg}[lau]{Algoritmi}
\newtheorem{esim}[lau]{Esimerkki}

\theoremstyle{remark}
\newtheorem*{huom}{Huomautus}


% --- tktltiki2 options ---
%
% The following commands define the information used to generate title and
% abstract pages. The following entries should be always specified:

\title{Metriikat ohjelmiston laadun mittaamiseen}
\author{Janne Haapsaari}
\date{\today}
\level{Aine}
\abstract{
  \begin{itemize}
    \item Johdanto
    \begin{itemize}
      \item Konteksti (kuka tarvitsee, mitä, miksi ongelma)
      \item Tutkimuskysymyksen ja -menetelmän asettaminen
      \item Tulosten yhteenveto
      \item Lukujen roolitus
    \end{itemize}
    \item Luku 2 - Ohjelmiston laatu?
      \begin{itemize}
        \item Konteksti ja käsitteet
        \item Haasteet ja arviointikriteerit
        \item Arvote
        \item Tutkimuskysymyksen analyysi
      \end{itemize}
    \item Luku 3 - Metriikat?
      \begin{itemize}
        \item Osaongelmien ratkaiseminen
        \item Yksityiskohtien arviointi ja rinnastus kirjallisuuteen
      \end{itemize}
    \item Luku N
      \begin{itemize}
        \item Evaluiointi eli johtopäätösten tekeminen
        \item Analysis, evaluation, experiences, recommendations, future improvement needs
      \end{itemize}
    \item Yhteenveto
      \begin{itemize}
        \item Mihin pyrittiin, miten ratkaistiin?
        \item Yhteenveto tuloksista ja niiden uskottavuudesta
        \item \textbf{Tulosten impakti}.
      \end{itemize}
  \end{itemize}
}

% The following can be used to specify keywords and classification of the paper:

\keywords{avainsana 1, avainsana 2, avainsana 3}

% classification according to ACM Computing Classification System (http://www.acm.org/about/class/)
% This is probably mostly relevant for computer scientists
% uncomment the following; contents of \classification will be printed under the abstract with a title
% "ACM Computing Classification System (CCS):"
% \classification{}

% If the automatic page number counting is not working as desired in your case,
% uncomment the following to manually set the number of pages displayed in the abstract page:
%
% \numberofpagesinformation{16 sivua + 10 sivua liitteissä}
%
% If you are not a computer scientist, you will want to uncomment the following by hand and specify
% your department, faculty and subject by hand:
%
% \faculty{Matemaattis-luonnontieteellinen}
% \department{Tietojenkäsittelytieteen laitos}
% \subject{Tietojenkäsittelytiede}
%
% If you are not from the University of Helsinki, then you will most likely want to set these also:
%
% \university{Helsingin Yliopisto}
% \universitylong{HELSINGIN YLIOPISTO --- HELSINGFORS UNIVERSITET --- UNIVERSITY OF HELSINKI} % displayed on the top of the abstract page
% \city{Helsinki}
%


\begin{document}

% --- Front matter ---

\frontmatter      % roman page numbering for front matter

\maketitle        % title page
\makeabstract     % abstract page

\tableofcontents  % table of contents

% --- Main matter ---

\mainmatter       % clear page, start arabic page numbering

\section{Johdanto}

Ohjelmisto on kokonaisuus, jonka muodostavat sovelluksen lähdekoodi, prosessit, dokumentaatio ja ohjelmiston käyttämä tieto, jotka liittyvät ja joita tarvitaan järjestelmän toimintaan~\cite{IEEE90, ISO9000}. Elinkaarensa aikana ohjelmiston on sopeuduttava muutoksiin, joita aiheutuu muun muassa uusien vaatimuksien, muuttuvan ympäristön ja korjausta vaativien virheiden seurauksena. Muutokset ohjelmistoon ovat väistämättömiä, ja siksi niihin tulee varautua. Ohjelmiston toiminnallisuuden toteuttamiseen on usein monia ratkaisumalleja eikä sopivimman ratkaisun valitseminen ole helppo tehtävä~\cite{CK94}. Varsinkin jälkikäteen tehtävät muutokset ovat usein vaikeita ja tekevät ohjelmistokehityksestä haastavaa.

Ohjelmiston koon kasvaessa se monimutkaistuu ja kykymme ymmärtää sen toteutusta laskee. Ohjelmistometriikoiden on tarkoitus auttaa kehittäjiä ymmärtämään paremmin ohjelmiston toimintaa sekä ohjata ja auttaa arvioimaan kehitysprosessia. Erilaisia ohjelmistometriikoita on esitetty kirjallisuudessa lukuisia~\cite{M76, H77, CK94, GKMS00, ZN08} ja niitä on tutkittu lähes koko ohjelmistoteollisuuden olemassaolon ajan. Ohjelmistometriikat ovat joukko automatisoitavia menetelmiä arvioimaan ohjelmiston laatua ja tuotantoprosessin toimivuutta. Ohjelmiston laadullinen arviointi on tuotantovaiheessa tärkeää sillä virheiden korjaaminen jälkikäteen on paitsi vaikeampaa myös kalliimpaa. Muutoksien seurauksena suuri osa ohjelmistoprojektien budjeteista kuluu ohjelmiston ylläpitovaiheeseen~\cite{CALO94, G83}. Myös ohjelmistoprojektin toimivuuden arvioiminen on tärkeää, jotta rajalliset resurssit kyetään käyttämään parhaalla mahdollisella tavalla.

Tämä tutkielma on kirjallisuuskatsaus ohjelmistoa mittaaviin metriikoihin. Ohjelmistoa mittaavat metriikat ovat perinteisesti olleet regressiopohjaisia ja pyrkineet muodostamaan arvioita virheiden määrästä ohjelmistossa sekä paikantamaan virheherkkiä ohjelmiston osa-alueita. Viime vuosina perinteisten metriikoiden rinnalle on tullut kokonaisvaltaisia metriikoita. Kokonaisvaltaiset metriikat rakentuvat perinteisten metriikoiden varaan, mutta ottavat lisäksi huomioon epävarmuustekijöitä ja syy-seuraussuhteita.

Tässä tutkielmassa esitellään erilaisia ohjelmistometriikoita ja niiden kykyä arvioida ohjelmiston laatua. Esittelen oliopohjaiseen suunnitteluun tarkoitetun metriikan~\cite{CK91, CK94}, ohjelmiston muutokseen perustuvan koodikirnun (code churn)~\cite{NB05} ja Bayes-verkkopohjaisen (Bayes Belief Network) metriikan~\cite{FNMHMKM07}. Valitut ohjelmistometriikat ovat kaikki toiminnaltaan erilaisia ja edustavat kattavasti ohjelmistometriikkoiden eri tyylisuuntia.

\newpage

\section{Ohjelmiston laadun arviointi}

Ohjelmiston laadun määrittäminen ei ole yksiselitteinen asia, sillä laatu merkitsee eri sidosryhmille erilaisia asioita. Määritelmiä ohjelmiston laadulle löytyy useita. IEEE määrittelee ohjelmiston laadun sen komponentin tai prosessin kykynä täyttää sille asetetut vaatimukset sekä asiakkaan tai käyttäjän sille asettamat vaatimukset ja odotukset~\cite{IEEE90}. Kansainvälinen ISO 9000 -standardi määrittelee ohjelmiston laadun sen ominaispiirteiden kokonaisuutena, jotka vaikuttavat ohjelmiston kykyyn täyttää sille asetettuja tai oletettuja tarpeita~\cite{ISO9000, MD06}. Juran muotoilee laadun vapaudeksi virheistä~\cite{JGB74}. Ohjelmiston laatu on kokonaisuus ohjelmiston ominaisuuksia, jotka vastaavat asiakkaan tarpeita ja siten mahdollistavat tuotetyytyväisyyden~\cite{JGB74}. Pressmanin mukaan laadukkaan ohjelmiston on noudatettava sille täsmällisesti määriteltyjä toiminnallisuus- ja tehokkuusvaatimuksia, eksplisiittisesti dokumentoituja käytäntöjä sekä vaatimuksia joita voidaan olettaa löytyvän ammattimaisesta sovelluksesta~\cite{PI92}.

Ohjelmiston laadulle on kansainvälinen standardi ISO 25010~\cite{ISO25010}. Standardi esittelee ohjelmiston laatumallin, joka koostuu ulkoisista ja sisäisistä laatutekijöistä. Laatumallin avulla voidaan määrittää ohjelmiston laadulliset kriteerit. Ulkoiset laatutekijät kuvaavat ohjelmiston toiminnallisuutta ja käyttäjäkokemusta. Ulkoisia laatutekijöitä ovat muun muassa toiminnallinen sopivuus, käytettävyys ja luotettavuus. Sisäiset laatutekijät kuvaavat ohjelmiston sisäistä toteutusta ja rakennetta. Sisäisiä laatutekijöitä ovat suorituskyky, turvallisuus, yhteensopivuus, siirrettävyys ja ylläpidettävyys. Korostaakseen käytettävyyden merkitystä ohjelmiston laadun arvioimisessa, standardissa määritellään erikseen ohjelmiston käytön laatu. Ohjelmiston käytön laatu koostuu toiminallisuudesta, toiminnallisuuden tehokkuudesta, käyttäjätyytyväisyydestä, käyttöturvallisuudesta ja käytettävyydestä.

Selvä yhtäläisyys monille ohjelmiston laadun määritelmille on asiakaslähtöisyys ja käyttäjätyytyväisyys. Laadukkaan ohjelmiston on paitsi suoriuduttava sille asetetuista toiminnallisuus- ja tehokkuusvaatimuksista, myös varauduttava muutoksiin. Mikäli muutoksiin ei osata varautua, saattaa edessä olla tilanne, jossa jokainen uusi muutos järjestelmään aiheuttaa keskimäärin yhden uuden virheen ohjelmistossa~\cite{LB85}. Tällaisessa tilanteessa ohjelmistoa on vaikea ylläpitää, eikä virheiden määrä ohjelmistossa enää laske. Lisäksi ohjelmistotuotantoprojekteissa on ulkoisia tekijöitä, kuten aikataulu, henkilöstöresurssit ja budjetti, jotka vaikuttavat välillisesti tuotettavan ohjelmiston laatuun.

Ohjelmistoteollisuudella on moneen muuhun teollisuuden alaan verrattuna huono maine projektien onnistumisen suhteen. Tutkimusten mukaan jopa 60-80\% ohjelmistoprojekteista ylittävät alkuperäiset arviot aikataulusta ja budjetista~\cite{MJ03}. Heikkoa tulosta mahdollisesti selittää se, että ohjelmistoprojektien arviointi on perustunut pitkälti asiantuntija-arvioihin. Tilanteen parantamiseksi kehitetty niin ohjelmistotuotantomenetelmiä kuin automatisoituja metriikoita ohjelmiston laadun ja monimutkaisuuden mittaamiseen. Ohjelmistotuotantomenetelmät jäävät tämän tutkielman aiheen ulkopuolelle.

\section{Ohjelmistometriikat}

[!!! Tähän tulee vielä Criterias for software metrics (Weuyker) !!!]\newline

Ohjelmistometriikoita on tutkittu koko sen ajan kun ohjelmistoja on tuotettu. Kattavasta tutkimuspohjasta huolimatta termi ohjelmistometriikka ei ole käsitteenä selkeä. Ohjelmistometriikasta on tullut yleinen nimitys laajalle joukolle mittaavia menetelmiä ohjelmistotuotannon parissa~\cite{FM00}. Joukkoon lukeutuvat ohjelmiston tunnusomaisia ominaisuuksia laskevat metriikat, tuotantoon vaadittavia resursseja ja ohjelmiston laatua arvioivat mallit ja ohjelmistosta löydettyjä virheitä dokumentoiva laadunvalvonnallinen puoli. Ohjelmistometriikat voidaan jakaa karkeasti itse ohjelmistoa (software product metrics) mittaaviin ja ohjelmistotuotantoprosessia (software process) mittaaviin metriikoihin~\cite{LH93}.

Kirjallisuudessa ensimmäiset ohjelmistometriikat esitettiin 1960-luvulla. Metriikat pohjautuivat ohjelmiston lähdekoodin fyysiseen kokoon. Martin esitti menetelmän, jossa laskettiin lähdekoodirivien (SLOC - Source Lines of Code) määrää~\cite{M65}. Halstead esitti menetelmän, joka laski operaattoreita ja operandeja lähdekoodista~\cite{H72, H77}. McCabe esitti kuuluisan syklomaattisen kompleksisuuden (Cyclomatic complexity) menetelmän, joka mittasi kontrollipolkujen määrää ohjelmistossa.~\cite{M76}. Ohjelmistometriikoilla pyrittiin arvioimaan tuottavuutta ja ohjelmiston toteutukseen ja testaukseen vaadittavaa työmäärää ja resursseja.

Ohjelmiston muutokseen perustuvat tekniikat ottavat hieman erilaisen lähtökohdan perinteisiin metriikoihin nähden. Ohjelmiston elinkaaren aikana kaikki siihen tehtävät muutokset eivät suinkaan ole seurausta korjatuista virheistä. Uusien ominaisuuksien lisääminen tai muuttuviin vaatimusmääreisiin vastaaminen, vaatii usein kokonaan uuden koodin lisäämistä ohjelmistoon. Lisäysten mukana ohjelmistoon saattaa tulla myös uusia virheitä. Hiljattain lisätty koodi on usein vähemmän testattua ja siten käytössä olleeseen koodiin verrattuna sisältää todennäköisemmin virheitä~\cite{ME98}.

\subsection{Oliopohjaiset metriikat}

Oliopohjaisen ohjelmistosuunnittelun yleistyessä 1980-luvun loppupuolella havaittiin, että olemassa olevat ohjelmistometriikat eivät ottaneet huomioon oliopohjaiselle suunnittelulle tyypillisiä ominaispiirteitä~\cite{LH93, CK94}. Tyypillisiä oliopohjaisen ohjelmoinnin piirteitä ovat oliot, luokat, perintä ja luokkien välinen kommunikointi viestien välityksellä. Näille piirteille ei ole olemassa vastaavia piirteitä proseduraalisessa ohjelmoinnissa, eikä siten myöskään käytössä olleissa ohjelmistometriikoissa. Tarvittiin siis ohjelmistometriikoita, jotka vastasivat oliopohjaisen suunnittelun tarpeisiin.

Chidamber ja Kemerer esittivät joukon olio-ohjelmointiin suunniteltuja ohjelmistometriikoita~\cite{CK91, CK94}. He kritisoivat että aikaisemmilta metriikoilta puuttuu täsmällinen teoreettinen ja matemaattinen pohja. Oliopohjaisen suunnittelun perustaksi he valitsivat Grady Boochin määritelmän~\cite{B94}. Määritelmän mukaan oliopohjaisessa suunnittelussa on neljä tärkeää askelta.
\begin{enumerate}
    \item Luokkien ja olioiden tunnistaminen (Identification of Classes), jossa ongelmakenttä mallinnetaan luokiksi ja olioiksi.
    \item Merkityksen tunnistaminen (Identify the Semantics), jossa määritellään edellisessä askeleessa muodostettujen luokkien elinkaari.
    \item Riippuvuuksien tunnistaminen (Indentify Relationships), jossa määritellään luokkien väliset riippuvuudet, kuten perintä ja näkyvyydet.
    \item Luokkien implementaatio (Implementation of Classes), jossa määritellään luokkien sisäinen rakenne.
\end{enumerate}

Chidamberin ja Kemererin oliopohjaiset ohjelmistometriikat koostuvat kuudesta ohjelmiston laatua arvioivasta mittarista.

\begin{enumerate}
    \item \textbf{Painotettu luokan metodien määrä} (Weighted Methods per Class, WMC). Metodien määrä ja monimutkaisuus toimivat ennusmerkkinä työmäärästä ja -ajasta, joka luokan kehitykseen ja ylläpitoon kuluu. Lisäksi suuri metodien määrä luokassa kasvattaa sen vaikutusta aliluokissa. Luokat, joissa on suuri määrä metodeja, ovat usein spesifisiä, ja tämä alentaa luokan uudelleenkäytettävyyttä.

    \item \textbf{Perimäpuun syvyys} (Depth of Inheritance Tree, DIT). Mitä syvemmällä luokka on perimäpuussa, sitä suurempi osa luokan metodeista on perittyjä. Syvä perimäpuu indikoi myös ohjelmiston suunnitelman monimutkaisuutta. Yliluokkien määrän kasvaessa lisääntyy myös riski, että jokin yliluokista aiheuttaa ongelmia luokalle.

    \item \textbf{Välittömien aliluokkien määrä} (Number of Children, NOC). Aliluokkien suuri määrä indikoi luokkien uudelleenkäyttöä. Luokan aliluokkien määrän kasvaessa myös riski perinnän väärinkäytöstä kasvaa. Suuri aliluokkien määrä saattaa myös indikoida sitä, että yliluokan abstraktiotaso on väärä. Välittömien aliluokkien määrä indikoi luokan vaikutusvallan määrää ohjelmiston suunnittelussa. Tällöin luokka vaatii usein kattavampaa testausta.

    \item \textbf{Luokkien välinen kytkeytyvyys} (Coupling between object classes, CBO). Luokan kytkeytyvyys mittaa niiden luokkien määrää, johon luokka on kytkeytynyt. Liiallinen luokkien välinen kytkeytyvyys on haitallista modulaariselle suunnittelulle ja heikentää luokkien uudelleenkäytettävyyttä. Luokkien välisen kytkeytyvyyden kasvaessa, luokan kapselointi (encapsulation) vähenee ja muutosten tekeminen vaikeutuu. Kytkeytyvyyden mittaaminen on avuksi määriteltäessä ohjelmiston osien testattavuutta.

    \item \textbf{Luokan vaste} (Response For a Class, RFC). Luokan vaste mittaa niiden luokan metodien määrää, joita luokan ilmentymä voi kutsua, kun jotakin luokan metodia kutsutaan. Mitä suurempi metodien määrä, sitä monimutkaisempi luokka on.

    \item \textbf{Metodien yhdenmukaisuuden puute} (Lack of Cohesion in Methods, LCOM). Metodien yhdenmukaisuuden puute mittaa metodiparien, jotka eivät jaa instanssimuuttujia, ja metodiparien, jotka jakavat instanssimuuttujia, välistä erotusta. Suureen arvo on nolla mikäli erotus on negatiivinen. Suureen suuri arvo indikoi sitä, että luokka yrittää tehdä liian montaa asiaa. Tällä on negatiivinen vaikutus luokan testattavuuteen.
\end{enumerate}

CK-metriikat on suunniteltu tarjoamaan analyyttisia ominaisuuksia ohjelmiston laadun arvioimiseen. Metriikoiden on tarkoitus olla kehittäjälle helposti omaksuttavia ja mahdollistaa mittauksen helppo automaattinen suorittaminen. Huomionarvoista on miten metriikkoja apuna käyttäen voi löytää suunnitteluvirheitä ja -rikkeitä jo ennen toteutusvaihetta~\cite{CK94}.

CK-metriikoiden on osoitettu sisältävän toivottuja ominaisuuksia oliopohjaisen ohjelmiston laadun ja monimutkaisuuden mittaamiseen. Metriikoiden on osoitettu ennustavan ohjelmiston virheherkkiä osa-alueita jo ohjelmiston suunnitteluvaiheessa.~\cite{CK94, BBM96, SK03}. Lisäksi CK-metriikoiden on osoitettu ennustavan paremmin ohjelmiston virheherkkyttä kuin perinteiset lähdekoodin fyysistä kokoa mittaavat metriikat~\cite{BB96}. Metriikoiden tehokkuuden on kuitenkin havaittu vaihtelevan eri ohjelmointikielillä ja -ympäristöissä~\cite{SK03}.


\subsection{Koodikirnu}

Koodikirnu on ohjelmistometriikka, joka mittaa ohjelmistoon tehtäviä muutoksia ja niiden laajuutta tietyllä aikavälillä~\cite{NB05}. Mittarin pohjana on oletus, että useasti muuttuva ohjelmiston osa sisältää enemmän virheitä kuin osa, joka muuttuu vähemmän samalla aikavälillä. Koodikirnun tarvitsemat tiedot muutoksista ohjelmistoon saadaan helposti ohjelmiston versionhallintajärjestelmästä.

Nagappanin ja Ballin esittelemä suhteellinen koodikirnu mittaa muutosten määrä ohjelmistossa tietyllä aikavälillä. Koodikirnu koostuu joukosta absoluuttisia mittareita.

\begin{enumerate}
    \item \textbf{Koodirivien kokonaismäärä}

      Ohjelmiston kaikkien lähdekooditiedostojen koodirivien yhteenlaskettu määrä.
    \item \textbf{Muokattujen koodirivien määrä}

      Tarkastellulla aikavälillä ohjelmistoon lisättyjen ja muokattujen koodirivien yhteenlaskettu määrä.
    \item \textbf{Poistettujen koodirivien määrä}

      Tarkastellulla aikavälillä ohjelmistosta poistettujen koodirivien yhteenlaskettu määrä.
    \item \textbf{Tiedostojen kokonaismäärä}

      Ohjelmiston lähdekooditiedostojen yhteenlaskettu määrä.
    \item \textbf{Muutoksien ajanjakso}

      Yhteen tiedostoon kohdistuneiden muutosten ajanjakson kesto.
    \item \textbf{Muutosten määrä}

      Tarkastellulla ajanjaksolla ohjelmiston lähdekooditiedostoihin tehtyjen muutosten yhteenlaskettu määrä.
    \item \textbf{Muokattujen tiedostojen kokonaismäärä}

      Ohjelmiston muokattujen lähdekooditiedostojen yhteenlaskettu määrä.
\end{enumerate}

Absoluuttisten mittarien pohjalle Nagappan ja Ball ovat kehittäneet joukon suhteellisia mittareita, joiden arvojen kasvun on osoitettu korreloivan virheiden määrän kasvun kanssa~\cite{NB05}.

\begin{enumerate}
    \item \textbf{Muokattujen koodirivien kokonaismäärä ja Koodirivien kokonaismäärä}

      Muokattujen koodirivien suuri määrä suhteessa koodirivien kokonaismäärään ennakoi suurta virhetiheyttä (defect density) ohjelmistossa.
    \item \textbf{Poistettujen koodirivien kokonaismäärä / Koodirivien kokonaismäärä}

      Poistettujen koodirivien suuri määrä suhteessa koodirivien kokonaismäärään ennakoi suurta virhetiheyttä ohjelmistossa.
    \item \textbf{Muokattujen tiedostojen kokonaismäärä / Tiedostojen kokonaismäärä}

      Muokattujen tiedostojen suuri määrä suhteessa tiedostojen kokonaismäärään kasvattaa todennäköisyyttä, että muokatut tiedostot lisäävät uusia virheitä ohjelmistoon.
    \item \textbf{Muutosten kokonaismäärä / Muokattujen tiedostojen kokonaismäärä}

      Usein muokatut tiedostot ovat todennäköisempiä lisäämään virheitä ohjelmistoon verrattuna tiedostoihin, joita muokataan harvemmin. Mittari on verrattavissa muokattujen tiedostojen määrän ja tiedostojen määrän suhteeseen.
    \item \textbf{Muutoksien ajanjakso / Tiedostojen kokonaismäärä}

      Mikäli muutoksen tekemiseen kuluva aika on suuri suhteessa tiedostojen määrään, kasvaa todennäköisyys että ohjelmiston tiedostot ovat monimutkaisia. Monimutkaisten tiedostojen muokkaaminen on vaikeaa ja lisää riskiä uusien virheiden lisäämisestä ohjelmistoon. Mittarin arvon kasvu indikoi kasvua ohjelmiston virhetiheydessä.
    \item \textbf{Muokattujen ja poistettujen koodirivien kokonaismäärä / Muutoksien ajanjakso}

      Muokattujen ja poistettujen koodirivien suuren määrän tulisi kasvattaa myös ajanjakson pituutta. Suuri mittarin arvo ennustaa suurempaa virhetiheyttä ohjelmistossa. Arvo on verrattavissa muutosten ajanjakson ja tiedostojen määrän suhteeseen.

    \item \textbf{Muokattujen koodirivien määrä / Poistettujen koodirivien määrä}

      Muokattujen koodirivien suuri määrä suhteessa poistettujen koodirivien määrään arvioi lisätyn koodin määrää ohjelmistossa. Lisättäessä ohjelmistoon uusia toiminnallisuuksia mittarin arvo kasvaa. Uusi koodi on usein vähemmän testattua ja siten saattaa sisältää enemmän virheitä. Arvo on vertailukelpoinen ensimmäisen ja toisen mittarin kanssa.
    \item \textbf{Muokattujen ja poistettujen koodirivien määrä / Muutosten määrä}

      Muokattujen ja poistettujen koodirivien ja muutosten määrän suhde mittaa todellisen muutoksen määrää ohjelmistossa. Mittarin suuri arvo ennustaa suurempaa virhetiheyttä ohjelmistossa.
\end{enumerate}

Koodikirnun tietojen keruu perustuu täysin versionhallintasovelluksen tarjoamaan versiotietoon, joten mittavirhe on pieni. Versiotietoihin vaikuttavat kehittäjien käytänteet. Jos kehittäjä tallentaa versionhallintajärjestelmään useita muutoksia yhdellä kertaa, saattaa osa muutoksista jäädä havaitsematta~\cite{NB05}. Lisäksi mitattavan ajanjakson pituus voi kasvaa merkittävästi kehittäjän toimesta. Myös tarkasteltavan ajanjakson pituus voi kasvaa, mikäli kehittäjä ei tallenna muutoksia välittömästi versionhallintajärjestelmään. Mittareiden vertailukelpoisuus vähentää käyttäjän toimista johtuvaa virhettä. Parhaiden mahdollisten tulosten saamiseksi vaaditaan kehittäjiltä sovittuja yhteisiä käytänteitä muutosten tallentamisessa versionhallintajärjestelmään.

Koodikirnun suhteellisten mittareiden on osoitettu olevan erinomaisia ennustamaan ohjelmiston virhetiheyttä~\cite{NB05}. Nagappan ja Bell tekevät koodikirnusta seuraavat johtopäätökset.

\begin{itemize}
    \item Suhteellisen koodikirnun mittareiden kasvu korreloi ohjelmiston virhetiheyden kasvun kanssa.
    \item Suhteelliset mittarit ovat parempia selittämään ohjelmiston virhetiheyttä kuin absoluuttiset mittarit.
    \item Suhteellinen koodikirnu toimii tehokkaana ennusteena ohjelmiston virhetiheydestä.
    \item Suhteellinen koodikirnu pystyy erottelemaan virheellisen ja toimivan ohjelmiston toisistaan.
\end{itemize}


\subsection{Bayes-verkot vai CACHED HISTORY}





\section{Ohjelmistometriikat laadun mittarina (Vertailu)}

Edellisessä kappaleessa esitettyjen ohjelmistometriikoiden lisäksi kirjallisuudessa on esitetty lukuisia muita, kuten verkkoanalyysiä ja koneoppimista apunaan käyttävät metriikat.


Tässä osiossa verrataan ja evaluoidaan esitettyjä ohjelmistometriikoita. Jonkinlaisia johtopäätöksiäkin olisi syytä muodostaa. Analyysia, evaluaatiota, kokemuksia, suositteluja, tulevaisuuden tutkimuskohteita\dots

Metriikat eivät toimi kovin hyvin yleisesti, vaan niiden tehokkuus vaihtelee eri ohjelmointikielillä. Tämän takia ohjelmistometriikat tulee valita käytössä olevan ohjelmointikielen ja -ympäristön mukaan.


\section{Yhteenveto}

Tässä osiossa tehdään yhteenveto tutkielmasta. Mihin pyrittiin, mitä ratkaistiin? Yhteenveto tuloksista ja niiden uskottavuudesta. Tulosten impakti?


Virheiden määrää ohjelmistossa on vaikeaa arvioida luotettavasti.

Ohjelmiston laatu ja sen arviointi ovat olleet eräitä ohjelmistoteollisuuden keskeisistä tutkimuskohteista.

Laadukas ohjelmisto vaatii, että kaikki neljä osa-aluetta on hyvin toteutettu. Ohjelmiston laadunvarmistuksen tulee ottaa huomioon ohjelmakoodin laadun lisäksi käytettyjen prosessien, dokumentaation ja tiedon laatu~\cite{G04}.

Ohjelmistometriikat tarjoavat määrällisen (quantitative) apuvälineen ymmärtämään ohjelmistotuotantoprosessin tehokkuutta ja laadullisten ongelma-alueiden paikantamiseen ohjelmistossa~\cite{PI92}.

Ohjelmistometriikat ovat johdon tukena määrällisessä ja laadullisessa päätöksenteossa, riskien hallinnassa ja minimoimisessa.

Vaikka ohjelmistometriikoista on pyritty tekemään yleisiä, on huomattu että metriikoiden luotettavuus vaihtelee eri ohjelmointikielillä.

Ohjelmistometriikoita kohtaan on osoitettu erilaista kritiikkiä. Esimerkiksi useat metriikoista vaativat muutoksia eri ohjelmointikielille.

% --- References ---
%
% bibtex is used to generate the bibliography. The babplain style
% will generate numeric references (e.g. [1]) appropriate for theoretical
% computer science. If you need alphanumeric references (e.g [Tur90]), use
%
% \bibliographystyle{babalpha-lf}
%
% instead.

\bibliographystyle{babplain-lf}
\bibliography{references-fi}


% --- Appendices ---

% uncomment the following

% \newpage
% \appendix
%
% \section{Esimerkkiliite}


\end{document}
